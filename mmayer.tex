%%%%%%%%%%%%%%%%%
% This is an example CV created using altacv.cls (v1.1.5, 1 December 2018) written by
% LianTze Lim (liantze@gmail.com), based on the
% Cv created by BusinessInsider at http://www.businessinsider.my/a-sample-resume-for-marissa-mayer-2016-7/?r=US&IR=T
%
%% It may be distributed and/or modified under the
%% conditions of the LaTeX Project Public License, either version 1.3
%% of this license or (at your option) any later version.
%% The latest version of this license is in
%%    http://www.latex-project.org/lppl.txt
%% and version 1.3 or later is part of all distributions of LaTeX
%% version 2003/12/01 or later.
%%%%%%%%%%%%%%%%

%% If you are using \orcid or academicons
%% icons, make sure you have the academicons
%% option here, and compile with XeLaTeX
%% or LuaLaTeX.
% \documentclass[10pt,a4paper,academicons]{altacv}

%% Use the "normalphoto" option if you want a normal photo instead of cropped to a circle
% \documentclass[10pt,a4paper,normalphoto]{altacv}

\documentclass[10pt,a4paper,ragged2e]{altacv}

%% AltaCV uses the fontawesome and academicon fonts
%% and packages.
%% See texdoc.net/pkg/fontawecome and http://texdoc.net/pkg/academicons for full list of symbols. You MUST compile with XeLaTeX or LuaLaTeX if you want to use academicons.

% Change the page layout if you need to
\geometry{left=1cm,right=9cm,marginparwidth=6.8cm,marginparsep=1.2cm,top=1.25cm,bottom=0.05 cm}

% Change the font if you want to, depending on whether
% you're using pdflatex or xelatex/lualatex
\ifxetexorluatex
  % If using xelatex or lualatex:
  \setmainfont{Carlito}
\else
  % If using pdflatex:
  \usepackage[utf8]{inputenc}
  \usepackage[T1]{fontenc}
  \usepackage[default]{lato}
  \usepackage{longtable}
  \usepackage{lastpage}
\fi

% Change the colours if you want to
\definecolor{VividPurple}{HTML}{3E0097}
\definecolor{SlateGrey}{HTML}{2E2E2E}
\definecolor{LightGrey}{HTML}{666666}
\colorlet{heading}{VividPurple}
\colorlet{accent}{VividPurple}
\colorlet{emphasis}{SlateGrey}
\colorlet{body}{LightGrey}

% Change the bullets for itemize and rating marker
% for \cvskill if you want to
\renewcommand{\itemmarker}{{\small\textbullet}}
\renewcommand{\ratingmarker}{\faCircle}

%% sample.bib contains your publications
\addbibresource{sample.bib}

\begin{document}
\name{Hyejeong Cheon}
\tagline{Who like biology and love mathematics}
% Cropped to square from https://en.wikipedia.org/wiki/Marissa_Mayer#/media/File:Marissa_Mayer_May_2014_(cropped).jpg, CC-BY 2.0
%\photo{2.5cm}{mmayer-wikipedia-cc-by-2_0}
\personalinfo{%
  % Not all of these are required!
  % You can add your own with \printinfo{symbol}{detail}
  \birth{28 Jul 1992}
  \email{hyejungchj@gmail.com}
  \phone{+47 486 86 196}
  %\mailaddress{}
  \location{Arnes bergsgårds veg 4 7033, Trondheim, Norway} \\
  \homepage{hyejeonc.github.io} 
  %\twitter{@marissamayer}
  \linkedin{www.linkedin.com/in/hyejeonc}
   \github{github.com/hyejeonc} % I'm just making this up though.
%   \orcid{orcid.org/0000-0000-0000-0000} % Obviously making this up too. If you want to use this field (and also other academicons symbols), add "academicons" option to \documentclass{altacv}
}

%% Make the header extend all the way to the right, if you want.
\begin{fullwidth}
\makecvheader
\end{fullwidth}

%% Depending on your tastes, you may want to make fonts of itemize environments slightly smaller
\AtBeginEnvironment{itemize}{\small}

%% Provide the file name containing the sidebar contents as an optional parameter to \cvsection.
%% You can always just use \marginpar{...} if you do
%% not need to align the top of the contents to any
%% \cvsection title in the "main" bar.
\cvsection[page1sidebar]{Work Experience}
\cvevent{Research engineer}{SK Hynix Semiconductor Inc.}{Jan 2015 -- July 2017}{Icheon, Republic of Korea}
\begin{itemize}
\item Studied deposition of dielectric materials and diffusion processes. 
\item Analyzed physical and chemical properties of layers by microscopy and spectroscopy. 
\item Set up ALD (Atomic Layer Deposition) machine and participated 48-stacked and 96-stacked NAND flash devices.


\end{itemize}

\divider%%%

\cvevent{Research internship}{Complex Biomaterials and Tissue Engineering Lab., Chung-ang Univ.}{June 2011 -- Dec 2013}{Seoul, Republic of Korea}
\begin{itemize}
\item Designed cell culture shell for biocompatible material with electric stimuli.
\item Synthesized Graphene layer and culturing cells on graphene substrate.
\end{itemize}

\cvsection[page1sidebar]{Education}
\cvevent{M. Sc. in Physics}{NTNU (Norwegian University of Science and Technology)}{Aug 2017 -- (present) 2019 }{Trondheim, Norway}

\begin{itemize}
\item Specialize in Biophysics and Bioinformatics.
\item Thesis “Monte Carlo simulation of halloysite nanotube : Study of polyelectrolyte conformation”
\end{itemize}

\divider%%%

\cvevent{Exchange student in Physics}{Umeå University}{Jan 2014 -- June 2014}{Umeå, Sweden}
\begin{itemize}
\item Enjoyed practical courses with lab sessions, such as Arctic science, Quantum physics, Fluid mechanics and Electromagnetics.
\end{itemize}

\divider%%%

\cvevent{B. Eng. in Nanobiomaterials}{Chung-Ang University}{Mar 2011 -- Feb 2015}{Seoul, Republic of Korea}
\begin{itemize}
\item Experienced various types of subjects in department of Integrative engineering and Natural science.
\item Thesis “Effect of electric stimuli to MC3T3-E1 cell on graphene substrate”
\end{itemize}

\cvsection[page1sidebar]{Certificate \& Publications}
\nocite{*}
\printbibliography[heading=pubtype,title={\printinfo{\faFileTextO}{Patent}}, type=patent]



\iffalse
\divider%%

\printbibliography[heading=pubtype,title={\printinfo{\faGroup}{Conference Proceedings}}, type=article]


\clearpage


\cvsection[page2sidebar]{Publications}

\nocite{*}

\printbibliography[heading=pubtype,title={\printinfo{\faBook}{Books}},type=book]

\divider

\printbibliography[heading=pubtype,title={\printinfo{\faFileTextO}{Journal Articles}}, type=article]

\divider

\printbibliography[heading=pubtype,title={\printinfo{\faGroup}{Conference Proceedings}},type=inproceedings]


\fi

%% If the NEXT page doesn't start with a \cvsection but you'd
%% still like to add a sidebar, then use this command on THIS
%% page to add it. The optional argument lets you pull up the
%% sidebar a bit so that it looks aligned with the top of the
%% main column.
% \addnextpagesidebar[-1ex]{page3sidebar}


\iffalse
MISCELLANEOUS
	Have technical skills in TEM, SEM and AFM 풀어서 쓰기
Work as an organizer in Likevektprogrammet Curie, NTNU
Volunteer as a hostess, UKA17
Work as a server in an Italian restaurant, BistroN 굳이 쓸 필요가 없다 근데 시간을 


\fi


\end{document}
